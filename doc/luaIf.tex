\documentclass{book}
\usepackage{verbatim}

\title{The Lua Interactive Fiction Module}
\author{Jeffrey Adair}

\newcommand{\myref}[1]{\ref{#1} starting on page \pageref{#1}}

\begin{document}
\maketitle

\chapter{Introduction}

The Lua Interactive Fiction Module (luaIf) provides the basis for
creating an interactive fiction game. The library is designed to make
reasonable assumptions about the world you describe, while allowing
you to override those assumptions where it's importaint for the game.

At the base level, the library provides a class system, parser, and
gramar description meachanism. Built on top of this base is a set of
grammar rules describing the natural interactions in the world.

\section{Conventions Used in This Book}

When describing the members of a ``class,'' an example instance of the
class matching the class' name but shown in lower case is used. For
example \emph{object:member()} referes to a function called on an
instance of the Object class--that is an instance created by calling
Object:new().

All classes in luaIf are named with capitalized names. When a function
or method of the class--that is the prototype table itself--is
referenced, the capitalized name is used as in the reference to
Object:new() above.

When referencing functions, a pair of parentesis are always included
even when the call list is omitted. In this way we ensure that it is
always clear that we are talking about a function and not a
variable. The parameter list is often ommitted for brevity when it it
is not required for completeness.

\section{Hello Tiny World}

Before we dive into the depths of creating a game using this library,
let's take a look at a clasic Hello World! example. Although it is
short, this game has points and an ending condition, and as such it
shows some of the features of the luaIf environment.

%\verbatiminput{helloworld.lua}

And here is an example of someone playing this game.

%\verbatiminput{helloworld.txt}

%%%%%%%%%%%%%%%%%%%%%%%%%%%%%%%%%%%%%%%%%%
\chapter{Writing Interactive Fiction}

\section{Describing the Scene}

When writing an interactive fiction game, the largest task is, of
course, describing the world in which it takes place. Every room must
be described to the computer, and for every object you must describe
both its apearance and its behavior.

\subsection{Hello Larger World}

\section{New Types}

\section{Grammar}

\subsection{Handling Events and Hooks}
\label{section:handlingEventsAndHooks}


%%%%%%%%%%%%%%%%%%%%%%%%%%%%%%%%%5%%%%%%%
\chapter{Technical Reference}

\section{Class System}

Lua does not provide an object-oriented programming environment, but
it does provide the means to create one within the language. The
aproach taken in luaIf is similar to the one described in
\textit{Programming in Lua 5th edition} by using metatables to tie
together instances and prototypes.

Objects are instantiated using the new() method on the desired
class. You can pass an initialized table to this method to initialize
member data or override functions, or you can pass nothing (or nil) to
construct a default object. The resulting object is returned from the
method. If a table is passed, it will be converted to an instance of
the class, overriding any existing class membership. The constructor
does not create a copy of the table. However, this behavior should not
be relied uppon: the calling function should always use the return
value from new() and not use existing references to the table passed
in.

\subsection{luaIf.Object}
\label{section:luaIf.Object}

The luaIf.Object object is the base class of all objects. It has the
following members:

\begin{description}
\item[object.pronoun] This variable contains the pronoun used to
  describe the object (the, a, some, etc.). This variable defaults to
  'the' if not set elsewhere. If the pronoun is capitalized, it will
  always apear capitalized, otherwise the capitalization will be
  adjusted accoding to use.
\item[object:isA(type)] This function returns true if the `type'
  parameter is one of the base classes of the object. Type can either
  be one of the class tables, or a string matching the type name. For
  example, ``Thing'' tests if the object is of type luaIf.Thing.
\item[object:parrent()] Returns the parrent type of an object. For an
  instance, the parrent() function returns the class prototype
  table. When called on class prototypes, this returns the
  supperclass.
\item[object:match(tokens)] tokens is a table of strings each of which
  is a token out of the parser. This function returns true when the
  all of the tokens in the given array are eitehr found in the
  luaIf.junkWords table or the object.name table. See
  section \myref{section:matchingCaptures} for details.
\item[object:placeIn(obj)] Causes obj to be contained by
  object. It is easy to confuse the syntax of this command because it
  reads as if you are placing object in obj, but in fact the opposite
  is true. This command calls obj:liberate() before placing the object
  in the object.contains array. This method also sets the
  obj.containedBy variable to object.
\item[object:placeOn(obj)] Causes obj to be supported by
  object. Again, be mindful of the reversed symantecs. Like the
  placeIn() method, this function calls obj:liberate() before placing
  obj in the object.supports array. This method also sets the
  obj.supportedBy variable to object.
\item[object:removeFrom(obj)] If obj is supported by object, it is
  removed from the object.supports array. If obj is contained by
  object, it is removed from the object.conains array. It is usualy
  better to call obj:liberate() instead.
\item[object:liberate()] Causes object to no longer be contained or supported.
\item[objet:details()] Returns a string containing the details of an
  object. Usualy returns the desc variable, but objects can override
  this method to provide programatic descriptions.
\item[object:phrase(pronoun, capitalize)] Returns a string containing
  the short name of an object. Usualy returns ``object.pronoun
  object.short'', but objects can override this method to provide
  programatic short names. If pronoun is false, the phrase should be
  returned without a pronoun. If capitalize is true, the resulting
  string is to be used at the beginging of a scentance.
\item[object:describe(contentsLevel, supportsLevel, curLevel)] Prints
  a description of the object based on teh object:phrase() and
  object:details() functions. contentsLevel indicates the number of
  recursions for printing the contents of the object. supportsLevel
  indicates the number of recursions for printing what the object
  supports. They default to 1 and 2 respectively. curLevel is a string
  appended to each line and is used to control indentation. It
  defaults to ``''.
\item[object:describeChildren(contentsLevel, supportsLevel, curLevel)]
  Like object:describe() except that the phrase and details are
  ommited.
\item[object:shortDescribe(pronoun, contentsLevel, supportsLevel,
  curLevel)] Like object:describe except that the details are
  ommited. pronoun determines the value passed to the object:phrase()
  function.
\item[object:before(what, ...)] Calls hook(``before'', what,
  ...). used to signal that the event ``what'' is about to happen to
  the object.
\item[object:after(what, ...)] Calls hook(``after'', what, ...). Used
  to signal that the event ``what'' has successfully happened to the object.
\item[object:alert(what, ...)] Calls hook(``alert'', what, ...). Used
  to signal that the event ``what'' is about to happen to annother object
  related to object. This usualy indicates soemthing happening to an
  object contained by object.
\item[object:hook(when, what, ...)] This is the main hook handler. It
  looks for functions to handle the specified when/what combination
  and calls it passing in the remaining parameters (...). To find a
  hook handler, this method constructs the string when..``\_''..what
  and uses it as a key on the current object. If a function exists
  with that name, it is called with the additional parameters to
  hook().

  See section \myref{section:handlingEventsAndHooks} for more details on hooks.
\end{description}

\subsection{Subtypes}

The following types are derived from luaIf.Object. They do not provide
any additional functionality over the base class except to
differentiate between types of object in the scene.

\begin{itemize}
\item Thing
\item Room
\item Person
\end{itemize}

\section{Parser}

Statement parsing is performed in three phases.

\begin{enumerate}
\item Break the input string into tokens
\item Match grammar rules
\item Match captures
\end{enumerate}

Each step is described in detail below.

\subsection{Tokenization}

The first step in parsing an input stream is to break it into
tokens. This is done by the luaIf.tokenize() function. This function
takes an input string and returns an array of strings built by
deconstructing the input into tokens.

To break the string into tokens, the tokenizer examines each character
to determine which class it is in. The character classes are:
alphanumeric characters plus the hiphen, punctuation, and others. Each
contiguous region of like-typed characters become a token. ``Other''
characters are dropped.

For example, the string ``Hi there, how are you?'' is tokenized into
\begin{itemize}
\item Hi
\item there
\item ,
\item how
\item are
\item you
\item ?
\end{itemize}

\subsection{Matching Grammar Rules}
\label{section:matchingGrammarRules}

For a complete description of how to write a grammar rule, see
section \myref{section:writingGrammarRules}.

First, the longest verb string that matches the start of the string is
found. This is as a key to find all grammar rules to check. The
remainder of the string is matched against the grammar rules
associated with that string to find a match. If none is found, then
the rest of the (shorter) verb strings are checked to try to find more
verbs. The luaIf.addVerb() function ``normalizes'' the grammar to only
have one verb token. (Multiple entries are added if multiple verbs
were specified.) This simplifies the matching of each grammar
specification.

For each grammar specification, the system tries to find all of the
literal tokens in the specification within the token list. The literal
tokens are used to partition the input tokens. The grammar must
specifiy a ``capture'' operation by including one of the values in the
luaIf.capture table for all tokens not consumed by the literals. These
are collected up for matching in the next step.

\begin{verbatim}
Tokens={"place", "the", "ball", "on", "the", "counter"};
Grammar={"place", luaIf.capture.Thing, "on", luaIf.capture.Thing};
\end{verbatim}

First, the verb ``place'' is consumed. Next a luaIf.capture.Thing is
encountered. Because it is a capture, tokens are collected until the
next literal ``on'' is encountered. Thus the first caputre collects
``the'' and ``ball.'' Then the ``on'' is consumed as a literal. The
parser then encounters annother capture and so begins collecting
tokens. Because there is not annother literal in the grammar
specification, the remaining tokens--``the'' and ``counter''--are
collected.

If the grammar were instead

\begin{verbatim}
Grammar={"place", luaIf.capture.Thing, "on", "top", "of",
luaIf.capture.Thing};
\end{verbatim}

\noindent then the grammar matching would fail when the ``top'' token
does not match the next available token after ``on.''

If a grammar specification matches, then processing continues with
capture matching. Otherwise, the grammar is rejected, and the next one
in the list is considered.

\subsection{Matching Captures}
\label{section:matchingCaptures}

At this point in the parsing, all of the literal tokens in a grammar
specification have been consumed, and there is a list of tokens
associated with each capture. The next step is to look in the sceen
for items that match the tokens. This process is called capture
matching.

For luaIf.capture.Thing, luaIf.capture.Room, etc., all of the objects
returned by the luaIf.visible() itterator are examined. The captured
tokens are passed to the object:match() function. If this function
returns true, then the object matches and takes the palce of the
captured tokens. The luaIf.visible() itterator recurses through the
sceen returning everything that can be seen in the current room--that
is contained by the room and not inside a closed non-transparent
container--and all visible things in the inventory.

The object:match() function as described in section
\ref{section:luaIf.Object} returns true if all of the tokens passed to
it are contained in either the luaIf.junkWords list, or in the
object's name list.

luaIf.capture.Inventory performs a similar search, however it only
examines items visible in the player's inventory.

luaIf.capture.Rest is a special kind of capture. Instead of finding an
object, this capture remains as a list of tokens when passed to the
handler function.

If no object can be found for one of the captures, then the grammar is
rejected, and the next grammar spec is considered as described in
section \ref{section:matchingGrammarRules}.


\end{document}
