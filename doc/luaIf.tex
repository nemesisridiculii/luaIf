\documentclass{book}
\title{The Lua Interactive Fiction Module}
\author{Jeffrey Adair}

\begin{document}
\maketitle

\chapter{Introduction}

The Lua Interactive Fiction Module (luaIf) provides the basis for
creating an interactive fiction game. The library is designed to make
reasonable assumptions about the world you describe, while allowing
you to override those assumptions where it's importaint for the game.

At the base level, the library provides a class system, parser, and
gramar description meachanism. Built on top of this base is a set of
grammar rules describing the natural interactions in the world.

\section{Class System}

Lua does not provide an object-oriented programming environment, but
it does provide the means to create one within the language. The
aproach taken in luaIf is similar to the one described in Programming
in Lua 5th edition by using metatables to tie together instances and
prototypes.

The luaIf.Object object is the base class of all objects. It has the
following members:

\begin{description}
\item[object:isA(type)] This function returns true if the `type'
  parameter is one of the base classes of the object. Type can either
  be one of the class tables, or a string matching the type name. For
  example, ``Thing'' tests if the object is of type luaIf.Thing.

\end{description}

\end{document}
